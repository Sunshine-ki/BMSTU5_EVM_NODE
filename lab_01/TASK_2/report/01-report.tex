\chapter{TASK\_1. Задание 1}

\section{Техническое задание:}

\begin{itemize}
	\item Создать хранилище в оперативной памяти для хранения информации о детях.
	\item Необходимо хранить информацию о ребенке: фамилия и возраст.
	\item Необходимо обеспечить уникальность фамилий детей.
\end{itemize}

\section{Реализовать функции:}

\begin{itemize}
	\item CREATE READ UPDATE DELETE для детей в хранилище
	\item Получение среднего возраста детей
	\item Получение информации о самом старшем ребенке
	\item Получение информации о детях, возраст которых входит в заданный отрезок
	\item Получение информации о детях, фамилия которых начинается с заданной буквы
	\item Получение информации о детях, фамилия которых длиннее заданного количества символов
	\item Получение информации о детях, фамилия которых начинается с гласной буквы
\end{itemize}

\begin{lstlisting}[caption=Код программы. Задание 1.]
"use strict";

class Kids {
	constructor() {
		this.arr = [];
	}

	add(surname, age) {
		// function func(elem) {
		// 	return elem.surname === surname;
		// }
		if (!this.arr.find(elem => elem.surname == surname)) {
			let new_kid = { surname, age }; // { surname: surname, age : age }
			this.arr.push(new_kid);
		}
	}

	log() {
		for (let i = 0; i < this.arr.length; i++)
			console.log(this.arr[i].surname, " ", this.arr[i].age);
		console.log("\n")
	}

	read(surname) {
		return this.arr.find(elem => elem.surname === surname);
		// for (let i = 0; i < this.arr.length; i++)
		// 	if (this.arr[i].surname === surname)
		// 		return this.arr[i];
		// return ""
	}

	update(surname, new_age, new_surname = false) {
		for (let i = 0; i < this.arr.length; i++) {
			if (this.arr[i].surname === surname) {
				this.arr[i].age = new_age;
				if (new_surname && !this.arr.find(elem => elem.surname == new_surname))
					this.arr[i].surname = new_surname;
				return;
			}
		}
	}

	delete(surname) {
		this.arr = this.arr.filter(elem =>
			elem.surname !== surname);
	}

	get_avg() {
		let sum = 0;
		let len = this.arr.length;
		for (let i = 0; i < len; i++)
			sum += this.arr[i].age;
		return len ? sum / len : 0;
	}

	get_eldest() {
		let len = this.arr.length;
		if (!len)
			return;
		let max = this.arr[0];
		for (let i = 1; i < len; i++) {
			if (this.arr[i].age > max.age)
				max = this.arr[i];
		}
		return max;
	}

	get_age_range(begin, end) {
		return this.arr.filter(elem => elem.age >= begin
			&& elem.age <= end);
	}

	get_surname_letter(letter) {
		return this.arr.filter(elem => elem.surname.charAt(0) === letter);
	}

	get_surname_len(len) {
		return this.arr.filter(elem => elem.surname.length > len)
	}

	get_surname_volwels() {
		const VOWELS = ['A', 'E', 'I', 'O', 'U'];
		return this.arr.filter(elem => VOWELS.indexOf(elem.surname.charAt(0).toUpperCase()) != -1)
	}
};
\end{lstlisting}


\begin{lstlisting}[caption=Код тестов. Задание 1.]
function test_kids() {
	let child = new Kids();
	child.add("Sukocheva", 2);
	child.add("Sukocheva", 3);
	child.add("Namestnik", 1);
	child.add("Vinogradov", 4);
	child.add("Volkov", 3);
	child.add("Orbitov", 7);


	child.log();

	console.log(child.read("Sukocheva"));
	child.update("Sukocheva", 4)
	console.log(child.read("Sukocheva"));

	child.log();
	child.delete("Vinogradov");
	child.log();

	child.read("Sukocheva")

	console.log(child.get_avg());
	console.log(child.get_eldest())

	console.log(child.get_age_range(1, 3))
	console.log(child.get_surname_letter('S'));

	console.log(child.get_surname_len(6));

	console.log(child.get_surname_volwels());
}
\end{lstlisting}


\chapter{TASK\_1. Задание 2}

\section{Техническое задание:}

\begin{itemize}
	\item Создать хранилище в оперативной памяти для хранения информации о студентах.
	\item Необходимо хранить информацию о студенте: название группы, номер студенческого билета, оценки по программированию.
	\item Необходимо обеспечить уникальность номеров студенческих билетов.
\end{itemize}

\section{Реализовать функции:}

\begin{itemize}
	\item CREATE READ UPDATE DELETE для студентов в хранилище
	\item Получение средней оценки заданного студента
	\item Получение информации о студентах в заданной группе
	\item Получение студента, у которого наибольшее количество оценок в заданной группе
	\item Получение студента, у которого нет оценок
\end{itemize}

\begin{lstlisting}[label=some-code,caption=Код программы. Задание 1.]
"use strict";

	class Students {
		constructor() {
			this.arr = [];
		}
	
		add(id, group, arr_score) {
			if (!this.arr.find(elem => elem.id === id)) {
				let new_student = { group, id, arr_score };
				this.arr.push(new_student);
			}
		}
	
		log() {
			console.log(this.arr);
		}
	
		read(id) {
			return this.arr.find(elem => elem.id === id);
		}
	
		update(id, group, arr_score) {
			let student = this.read(id);
			if (student) {
				student.group = group;
				student.arr_score = arr_score;
			}
		}
	
		delete(id) {
			this.arr = this.arr.filter(elem =>
				elem.id !== id);
		}
	
		get_avg(id) {
			let student = this.read(id);
			if (!student)
				return;
	
			let len = student.arr_score.length;
			let sum = 0;
			for (let i = 0; i < len; i++)
				sum += student.arr_score[i];
	
			return len ? sum / len : 0;
		}
	
		get_info_group(group) {
			return this.arr.filter(elem => elem.group === group);
		}
	
		get_student_max_count_score(group) {
			let students = this.get_info_group(group);
			if (!students)
				return;
			let max = students[0];
			for (let i = 1; i < students.length; i++) {
				if (students[i].arr_score.length > max.arr_score.length)
					max = students[i];
			}
			return max
		}
	
		get_student_no_score() {
			return this.arr.filter(elem => elem.arr_score.length === 0);
		}
	};
\end{lstlisting}

\begin{lstlisting}[caption=Код тестов. Задание 2.]
	function test_students() {
		let students = new Students();
	
		students.add(123456, "UI7-43b", [5, 5, 5]);
		students.add(123456, "UI7-43b", [5, 5, 5]);
	
		students.add(123457, "UI7-43b", []);
		students.add(123451, "UI7-43b", [1, 5, 5, 2]);
		students.add(123000, "UI7-44b", [4, 3, 4]);
		students.add(123444, "UI7-45b", [5, 4, 5]);
		students.add(123442, "UI7-45b", [2, 3, 3]);
		students.add(123441, "UI7-42b", [2, 3, 5]);
		students.add(123443, "UI7-41b", [5, 2, 3]);
	
		students.log();
	
		console.log(students.read(123456));
	
		students.update(123441, "UI7-42b", [5, 5, 5]);
		students.log();
	
		students.delete(123442);
		students.log();
	
		console.log(students.get_avg(123441));
		console.log(students.get_info_group("UI7-43b"));
	
		console.log(students.get_student_max_count_score("UI7-43b"))
		console.log(students.get_student_no_score());
	}
\end{lstlisting}


\chapter{TASK\_1. Задание 3}

\section{Техническое задание:}

\begin{itemize}
	\item Создать хранилище в оперативной памяти для хранения точек.
	\item Неоходимо хранить информацию о точке: имя точки, позиция X и позиция Y.
	\item Необходимо обеспечить уникальность имен точек.
\end{itemize}

\section{Реализовать функции:}

\begin{itemize}
	\item CREATE READ UPDATE DELETE для точек в хранилище
	\item Получение двух точек, между которыми наибольшее расстояние
	\item Получение точек, находящихся от заданной точки на расстоянии, не превышающем заданную константу
	\item Получение точек, находящихся выше / ниже / правее / левее заданной оси координат
	\item Получение точек, входящих внутрь заданной прямоугольной зоны
\end{itemize}


\begin{lstlisting}[label=some-code,caption=Код программы. Задание 3]
"use strict";

	class Points {
		constructor() {
			this.arr = [];
		}
	
		add(name_point, x, y) {
			if (!this.arr.find(elem => elem.name_point === name_point)) {
				let new_point = { name_point, x, y };
				this.arr.push(new_point);
			}
		}
	
		log() {
			console.log(this.arr);
		}
	
		read(name_point) {
			return this.arr.find(elem => elem.name_point === name_point);
		}
	
		update(name_point, x, y) {
			let point = this.read(name_point);
			if (point) {
				point.x = x;
				point.y = y;
			}
		}
	
		delete(name_point) {
			this.arr = this.arr.filter(elem =>
				elem.name_point !== name_point);
		}
	
		get_distance(p1, p2) {
			let dx = p1.x - p2.x;
			let dy = p1.y - p2.y;
			return Math.sqrt(dx * dx + dy * dy);
		}
	
		min_distance() {
			if (this.arr.length < 2)
				return;
	
			let len = this.arr.length;
			let p1 = this.arr[0];
			let p2 = this.arr[1];
			let min_dist = this.get_distance(p1, p2), current_dist;
			console.log(min_dist);
	
			for (let i = 0; i < len - 1; i++)
				for (let j = i + 1; j < len; j++) {
					current_dist = this.get_distance(this.arr[i], this.arr[j]);
					if (current_dist < min_dist)
						min_dist = current_dist;
				}
	
			return min_dist;
		}
	
		// Получение точек, находящихся от заданной точки на расстоянии,
		// не превышающем заданную константу
		get_points_less(point, max_len) {
			return this.arr.filter(elem =>
				this.get_distance(point, elem) <= max_len);
		}
	
		// axis: 0-x, 1-y;
		// direction: 0-'+', 1-'-';  
		get_points_axis(axis, direction) {
			let func;
	
			if (!axis && !direction)
				func = p => p.x > 0;
			else if (!axis && direction)
				func = p => p.x < 0;
			else if (!direction)
				func = p => p.y > 0;
			else
				func = p => p.y < 0;
	
			return this.arr.filter(func);
		}
		get_points_inside_rectangle(min_x, max_x, min_y, max_y) {
			return this.arr.filter(p =>
				p.x > min_x && p.x < max_x &&
				p.y > min_y && p.y < max_y);
		}
	}
	
\end{lstlisting}


\begin{lstlisting}[caption=Код тестов. Задание 3.]
	function test_points() {
		let points = new Points();
		points.add("p", 0, 0);
		points.add("p0", 3, 4);
		points.add("p1", 1, 1);
		points.add("p2", 10, 10);
		points.add("p3", 1, 10);
		points.add("p4", 10, 1);
		points.add("p5", 12, 0);
		points.add("p6", -12, 1);
		points.add("p7", 12, -1);
	
		points.log();
	
		console.log(points.read("p1"));
		console.log(points.read("p3"));
		console.log(points.read("p13"));
		console.log(points.read("p5"));
	
		points.update("p5", 100, 12);
		points.log();
		points.delete("p5")
		points.log();
	
		console.log(points.min_distance());
	
		let p = points.read("p"); // 0, 0
		console.log(points.get_points_less(p, 5));
	
		console.log("Points:\n")
		points.log();
		console.log("axis X-:\n", points.get_points_axis(0, 1));
		console.log("axis Y-:\n", points.get_points_axis(1, 1));
	
		console.log("rectangle: -10, 10, -10, 10\n", points.get_points_inside_rectangle(-10, 10, -10, 10));
	}	
\end{lstlisting}

\chapter{TASK\_2. Задание 1}

\section{Техническое задание:}

\begin{itemize}
	\item Создать класс Точка.
	\item Добавить классу точка Точка метод инициализации полей и метод вывода полей на экран
	\item Создать класс Отрезок.
	\item У класса Отрезок должны быть поля, являющиеся экземплярами класса Точка.
	\item Добавить классу Отрезок метод инициализации полей, метод вывода информации о полях на экран, а так же метод получения длины отрезка.
\end{itemize}


\begin{lstlisting}[caption=Код программы. Задание 1.]
	class Point {
		constructor(x, y) {
			this.set_data(x, y);
		}
	
		set_data(x, y) {
			this.x = x;
			this.y = y;
		}
	
		log() {
			console.log("X: ", this.x, "Y: ", this.y);
		}
	}
	
	class Line {
		constructor(x1, y1, x2, y2) {
			this.set_data(x1, y1, x2, y2);
		}
	
		set_data(x1, y1, x2, y2) {
			this.start_point = new Point(x1, y1);
			this.end_point = new Point(x2, y2);
		}
	
		get_distance() {
			const dx = this.start_point.x - this.end_point.x;
			const dy = this.start_point.y - this.end_point.y;
			return Math.sqrt(dx * dx + dy * dy);
		}
	
		log() {
			console.log("Начальная точка:");
			this.start_point.log();
			console.log("Конечная точка:");
			this.end_point.log();
		}
	}	
\end{lstlisting}

\chapter{TASK\_2. Задание 2}

\section{Техническое задание:}

\begin{itemize}
	\item Создать класс Треугольник.
	\item Класс Треугольник должен иметь поля, хранящие длины сторон треугольника.
	\item Необходимо обеспечить уникальность номеров студенческих билетов.
\end{itemize}

\section{Реализовать следующие методы:}

\begin{itemize}
	\item Метод инициализации полей
	\item Метод проверки возможности существования треугольника с такими сторонами
	\item Метод получения периметра треугольника
	\item Метод получения площади треугольника
	\item Метод для проверки факта: является ли треугольник прямоугольным
\end{itemize}


\begin{lstlisting}[label=some-code,caption=Код программы. Задание 1.]
	class Triangle {
		constructor(a, b, c) {
			this.set_data(a, b, c);
		}
	
		set_data(a, b, c) {
			this.a = a;
			this.b = b;
			this.c = c;
		}
	
		check_existence() {
			if ((this.a < this.b + this.c) &&
				(this.b < this.a + this.c) &&
				(this.c < this.b + this.a))
				return true;
			return false;
		}
	
		get_perimeter() {
			if (!this.check_existence())
				return;
			return this.a + this.b + this.c;
		}
	
		area() {
			if (!this.check_existence())
				return;
			let semi_perimeter = this.get_perimeter() / 2;
			return Math.sqrt(semi_perimeter * (semi_perimeter - this.a) *
				(semi_perimeter - this.b) * (semi_perimeter - this.c));
		}
	
		check_rectangular() {
			let EPS = 1e-5;
			if (!this.check_existence())
				return;
			let a = this.a, b = this.b, c = this.c;
	
			if ((a * a + b * b - c * c < EPS) ||
				(a * a + c * c - b * b < EPS) ||
				(c * c + b * b - a * a < EPS))
				return true;
			return false;
		}
	}
	
\end{lstlisting}


\chapter{TASK\_2. Задание 3}

\section{Техническое задание:}

\begin{itemize}
	\item Реализовать программу, в которой происходят следующие действия:
	\item Происходит вывод целых чисел от 1 до 10 с задержками в 2 секунды.
	\item После этого происходит вывод от 11 до 20 с задержками в 1 секунду.
	\item Потом опять происходит вывод чисел от 1 до 10 с задержками в 2 секунды.
	\item После этого происходит вывод от 11 до 20 с задержками в 1 секунду.
	\item Это должно происходить циклически.
\end{itemize}



\begin{lstlisting}[label=some-code,caption=Код программы. Задание 3]
	"use strict"

	const TIME_FIRST = 1000
	const TIME_SECOND = 500
	
	function timer() {
		let a = 1;
	
		let funcFirst = () => {
			console.log(a++);
			if (a > 10) {
				clearInterval(interval1);
				interval2 = setInterval(funcSecond, TIME_SECOND)
			}
		}
	
		let funcSecond = () => {
			console.log(a++);
			if (a > 20) {
				clearInterval(interval2);
				a = 1;
				interval1 = setInterval(funcFirst, TIME_FIRST)
			}
		}
	
		let interval1 = setInterval(funcFirst, TIME_FIRST);
		let interval2;
	}
	
\end{lstlisting}

